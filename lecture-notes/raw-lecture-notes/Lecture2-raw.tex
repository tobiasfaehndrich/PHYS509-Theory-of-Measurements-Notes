\section[Lecture2]{\hyperlink{toc}{Lecture2}}

Thursday, September 11th 2025

Bayes Formula
\begin{itemize}
      \item Let E, F be events
            \[ E = EF \cup EF^c \]
            \[ P(E) = P(EF) + P(EF^c)\]
            \[ P(E) = P(E|F)P(F) + P(E|F^c)P(F^c)\]
            \[ P(E) = P(E|F)P(F) + P(E|F^c)(1 - P(F))\]
      \item Example: Sp,e blood test 95\% effective in detecting a disease if in fact the person has the disease. It also has a "false positive" result s.t. gives a positive for 1\% of healthy people. Suppose 0.5\% of population has disease.

            \[ D = \text{person has disease}\]
            \[ E = \text{test is positive}\]

      \item We want
            \[ P(D|E) = \frac{P(ED)}{P(E)}\]
            \[ P(D|E) = \frac{P(E|D)P(D)}{P(E|D)P(D) + P(E|D^c)(1 - P(D))}\]
            \[ = \frac{0.95 \times 0.005}{0.95 \times 0.005 + 0.01 \times 0.995} = 0.32\]
      \item So even with a positive test, only 32\% chance of having the disease
      \item Law of Total Probability:
      \item Let $\{F_i\}$ be mutually exclusive events s.t. :
            \[ \cup^n_{i=1} F_i = S\]
            For any event E
            \[ E = E \cap (\cup^n_{i=1} F_i) = \cup^n_{i=1} (E F_i)\]
            \[ P(E) = P(\cup E F_i) = \sum_{i=1}^n P(E F_i) = \sum_{i=1}^n P(E|F_i)P(F_i)\]

      \item Idependent Events:

            Generally P(E|F) $\neq$ P(E)

            But if knowing F does not change prob e

            if

            \[ P(E|F) = \frac{P(EF)}{P(F)} = P(E)\]

            \[ \boxed{P(EF) = P(E)P(F)} \]


      \item E.g. roll 2 dice

            \[ E_1 \equiv \text{sum} = 6 \]
            \[ F \equiv \text{1st die} = 4 \]
            \[ E_1: \{(1,5), (2,4), (3,3), (4,2), (5,1)\} \]
            \[ F: \{(4,1), (4,2), (4,3), (4,4), (4,5), (4,6)\} \]
            \[ E_1 F = \{(4,2)\} \]
            \[ P(E_1 F) = \frac{1}{36} \]
            \[ P(E_1) = \frac{5}{36} \]
            \[ P(F) = \frac{6}{36}  = \frac{1}{6}\]
            \[ P(E_1 ) P(F) = \frac{5}{36} \times \frac{1}{6} = \frac{5}{216} \neq P(E_1 F)\]

            \[ E_2 \equiv \text{sum} = 7 \]
            \[ E_2: \{(1,6), (2,5), (3,4), (4,3), (5,2), (6,1)\} \]
            \[ E_2 F = \{(4,3)\} \]
            \[ P(E_2 ) = \frac{6}{36}  = \frac{1}{6}\]
            \[ P(F) = \frac{1}{6}\]
            \[ P(E_2 F) = \frac{1}{36} \]

      \item $S = \{ all possible outcomes of stochastic process, X\} $
            \[ x = \text{random variable} \]
            \[ S = \text{finite or countable infinite: discrete random variable}\]
            \[ S = \text{uncountable infinite: continuous random variable}\]

            \[ P(x_0, x_0 + dx) = \text{prob that if we take a trial of x, x is between } x_0 + dx\]
            \[ = p(x) d(x) (= f(x) dx \text{ in some books})\]
            \[ p(x) = \text{probability density function (pdf)}\]
      \item Discrete
            \[ S = S_i \]
            \[ p_i = \text{probability of } S_i , probability mass function (pmf)\]

            \[ 0 \leq P(S_i) \leq 1\]
            \[ 1 = P(s) \]
            \[ 0 \leq p(x) \nleq 1 \]
            \[ \int_0^b p(x) dx \leq 1 \]
            \[ \int_{-\infty}^{\infty} p(x) dx = 1 \]

      \item e.g. I manufactor resistors make a batch of 1k$\Omega$ resistors. I can (in principle) figure out the distribution of x.

      \item To describe p(x) in general we can't give $\infty$ information, so we specify
            \begin{itemize}
                  \item mode $\equiv$ peak value of p(x)
                  \item median $\equiv$ 50\% comulative-value
                  \item mean $\equiv$ average value of x weighted by p(x)
            \end{itemize}

      \item Comulative Distribution Function (cdf) here we call

            \[ F(x) = \int_{-\infty}^{x} p(x') dx' = P(X \leq x) \]
            \[ F(-\infty) = 0, F(\infty) = 1\]

      \item Expectation Value (mean):
      \item The expectation value of any function f(x) over p(x) if called E(f)
            \[ E(f) = \int_{\Omega} f(x) p(x) dx \]
            \[ E = \text{linear operator}\]
            \[ E(af +bg) = aE(f) + bE(g)\]

      \item Expectation of powers of x:
            \[ E(x^0) = E(1) = \int 1 p(x) dx = 1\]
            \[ E(x^1) = \int x p(x) dx \equiv \mu = \text{mean value of x}\]
            \[ E(x^2) = \int x^2 p(x) dx \equiv \sigma^2 = \text{variance of x}\]

      \item Imagine having all moments $E(x^n)$ for n = 0, 1, 2, ...
            \[ \int x^n p(x) dx \text{ for } n = 0, 1, 2, ...\]

      \item Characteristic Function of p(x):
            \[ \phi(t) = \int_{-\infty}^{\infty} e^{itx} p(x) dx \]
            \[ = \text{Fourier Transform of p(x)}\]
            \[ = E(e^{itx})\]
            \[ \phi(t) = E(e^{itx}) \]
            \[ = E (1 + i t x + \frac{(itx)^2}{2!} + ... ) \]
            \[ = 1 + it E(x) + \frac{(it)^2}{2!} E(x^2) + ... \]

            \[ = \sum_{k=0}^{\infty} \frac{(it)^k}{k!} \mu_{k'} \]

            \[ \mu_{k'} = k^n \text{moments} \]

      \item Given, $\phi(t)$ we can get p(x) by inverse Fourier transform:

            \[ \frac{d^n \phi(t)}{dt^n} \left| \right._{t=0} = i^n \mu_{n'} \]

      \item missed something here \dots

      \item Central moments:

            \[ E((x-\mu)^n) = \int (x-\mu)^n p(x) dx \equiv \mu_n \]
            \[ \mu = \text{mean value of x} = E(x) \]
      \item 1st centeral moment
            \[ E((x-\mu)^1) = E(x) - E(\mu) = \mu - \mu = 0\]
      \item 2nd central moment
            \[ E((x-\mu)^2) \equiv V(x) = \text{variance of x} = \sigma^2 \]
      \item 3rd
            \[ E((x-\mu)^3) = \text{skewness of x} \equiv \frac{E((x-\mu)^3)}{\sigma^3} \]
      \item 4th
            \[ E((x-\mu)^4) = \text{kurtosis of x} \equiv \frac{E((x-\mu)^4)}{\sigma^4} - 3 \]

            (-3 so that kurtosis of normal distribution is 0)
\end{itemize}

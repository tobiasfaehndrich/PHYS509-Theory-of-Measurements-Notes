\section[Lecture6]{\hyperlink{toc}{Lecture 6}}

Tuesday, September 25th 2025

\begin{itemize}
    \item Linear x-formation:
    \[ \vec{y} = A\vec{x} \]
    \[ V_{kl}(\vec{y}) = \sum_{i,j} \pdv{y_k}{x_i} \pdv{y_l}{x_j} V_{ij}(\vec{x}) \]
    \item Linear $y_k = \sum A_{kj} x_j$
    \item then
    \[ V_{kl}(\vec{y}) = \sum_{i,j} A_{ki} A_{lj} V_{ij}(\vec{x}) \]
    \item or in matrix form
    \[ V(\vec{y}) = \left(A V(\vec{x}) A^T\right)_{kl} \]
    \item If $\hat{e}_i$ are the eigenvectors of V, then
    \[ V (\vec{x})\hat{e}_i = \lambda_i \hat{e}_i \]
    \item Form:
    \[ A = \begin{pmatrix}
        \hat{e}_1\\
        \ldots\\
        \hat{e}_n
    \end{pmatrix}
    = \begin{pmatrix}
        \hat{e}_11 & \hat{e}_12 & \ldots & \hat{e}_1n\\
        \ldots & \ldots & \ldots & \ldots\\
        \hat{e}_n1 & \hat{e}_n2 & \ldots & \hat{e}_nn
    \end{pmatrix} \]
    \item then
    \[ A^T A = I \]
    \item then:
    \[ V A^T = V \begin{pmatrix}
        \hat{e}_1 & \ldots \\
        \ldots & \ldots \\
        \hat{e}_n & \ldots
    \end{pmatrix} = \begin{pmatrix}
        \lambda_1 \hat{e}_{11} & \ldots & \lambda_n \hat{e}_{n1} \\
        \ldots & \ldots & \ldots \\
        \lambda_1 \hat{e}_{1n} & \ldots & \lambda_n \hat{e}_{nn}
    \end{pmatrix}\]

    \item Then:
    
    \[ A V A^T = \begin{pmatrix}
        \hat{e}_{11} & \ldots & \hat{e}_{1n} \\
        \ldots & \ldots & \ldots 
    \end{pmatrix}
    \begin{pmatrix}
        \lambda_1 \hat{e}_{11} & \ldots  \\
        \ldots & \ldots & \ldots \\
        \lambda_1 \hat{e}_{1n} & \ldots
    \end{pmatrix}
    = \begin{pmatrix}
        \lambda_1 & 0 & \ldots & 0 \\
        0 & \lambda_2 & \ldots & 0 \\
        \ldots & \ldots & \ldots & \ldots \\
        0 & 0 & \ldots & \lambda_n
    \end{pmatrix} \]

    \item Then:

    \[ AVA^T = V(\vec{y}) = \begin{pmatrix}
        \sigma_1^2 & 0 & \ldots & 0 \\
        0 & \sigma_2^2 & \ldots & 0 \\
        \ldots & \ldots & \ldots & \ldots \\
        0 & 0 & \ldots & \sigma_n^2
    \end{pmatrix} \]
    \item Binomial distribution:
    \item Consider an experiment with two outcoes.
    \item E.g. coin flips, select a ball with 2 possible colours, etc.
    \item One trial is called a Bernoulli trial.
    \item Example -- Method 1: You have an urn filled with N balls. Some are red (R), some are blue (B).
    \item (0) What is your estimate of $n_R$, $n_B$, or f = $n_r/N$ or p of drawing R? 
    \item (1) You pick a ball: R. Q estimate of $p=n_r/N$?
    \item (2) You pick another without replacing 1st ball: get R.
    \item (3) R 
    \item (4) Get B 
    \item This is a question about this ONE urn.
    
    \item Now Method 2: you draw red, and you PUT IT BACK. You repeat this several times.
    \item Now Method 3: We have an infinite source of ball with fraction p of red balls and (1-p) of blue balls.
    \begin{align*}
        P(R) &= p \\
        P(B) &= 1-p
    \end{align*}
    \item Make infinite number urns all with N balls, with fraction p red and (1-p) blue.
    \item Open all count $n_R$ red balls, $n_B$ blue balls.
    \item In our case we have N balls, prob $p=R$ and $1-p=q=B$.
    \item Prob of getting sequence RRB is:
    \[ P(RRB) = p \cdot p \cdot (1-p) = p^2 (1-p) \]
    \item If we don't care about order, then:
    \[ P(RRB) = P(RBR) = P(BRR) = p^2 (1-p) \]
    \item There are 3 ways of ordering RRB, so total probability is:
    \[ P(2R,1B) = 3 p^2 (1-p) = 3 p^2 q \]
    \item Number of ways to choose r items from N is:
    \[ \binom{N}{r} = \frac{N!}{r!(N-r)!} \]
    \item Probability of getting exactly r R out of N:
    \[ P_r = \binom{N}{r} p^r (1-p)^{N-r} = B(r;N,p)\]
    \item This is called the Binomial distribution and applies to anything where there is 2 outcomes ($A, \bar{A}$)
    \item Want mean, sigma
    \begin{align*}
        E(r) &= \sum_{r=0}^n r P_r = \sum_{r=0}^n r \binom{n}{r} p^r (1-p)^{n-r} \\
        &= \sum_{r=0}^n r \frac{n!}{r!(n-r)!} p^r (1-p)^{n-r} \\
        &= \sum_{r=1}^n \frac{n!}{(r-1)!(n-r)!} p^r (1-p)^{n-r} \\
        &= np \sum_{r=1}^n \frac{(n-1)!}{(r-1)!(n-r)!} p^{r-1} (1-p)^{n-r}
    \end{align*}
    \item Change sum $r' = r-1$ $\rightarrow n'=n-1$
    \begin{align*}
        E(r) &= np \sum_{r'=0}^{n-1} \frac{(n-1)!}{r'!(n-1-r')!} p^{r'} (1-p)^{(n-1)-r'} \\
        &= np \sum_{r'=0}^{n-1} \binom{n-1}{r'} p^{r'} (1-p)^{n'-r'} \\
        &= np \cdot 1 = np
    \end{align*}

    \item from:
    \[ (p+q)^n = \sum_{r=0}^n \binom{n}{r} p^r q^{n-r} \]
    \[ (p+1-q)^n = 1^n = 1 \]
    \[ E(r) = np \]
    \item This is what we want!
    \item Now:
    
    \[ V(r) = \sum r^2 p_r -E(r)^2 = \sum r^2 p_r - n^2 p^2 \]
    \item Slightly easier to calc:
    \[ \sum r(r-1) p_r = \sum_{r=0}^n r(r-1) \frac{n!}{r!(n-r)!} p^r (1-p)^{n-r} = \sum_{r=2}^n \frac{n!}{(r-2)!(n-r)!} p^r q^{n-r} \]
    \[ = n(n-1) p^2 \sum_{r=2}^n \frac{(n-2)!}{(r-2)!(n-r)!} p^{r-2} q^{n-r} \]
    \item Sub $r' = r-2$
    \[ = n(n-1) p^2 \sum_{r'=0}^{n-2} \binom{n-2}{r'} p^{r'} q^{(n-2)-r'} \]
    \[ = n(n-1) p^2 \sum_{r'=0}^{n'} = \binom{n'}{r'} p^{r'} q^{n'-r'} \]
    \item where $n' = n-2$
    \[ = n(n-1)p^2 \cdot 1 = n(n-1)p^2 \]
    \[ = n^2 p^2 - np^2 \]
    \item Such that:
    
    \[ \boxed{V(r) = \sum r^2 p_r - n^2 p^2 }= np ( 1 - p) = npq \]

    \item Why is this important? Histograms are often Binomially distributed.
    \item Data either falls A: falls in bin, or $\bar{A}$ does not fall in bin.
    \item p = prob of falling in ith bin.
    \item +n entries, e.g. students in class, hist = grades
    \item Expected number of entries is np
    \item Plot of taking distribution several times and checking how many fall in bin i and then plotting that distribution is Binomial.
    \item Usually you have 1 histogram,
    \item Look at entries in bin i -- $n_i/n$ = fraction of entries in bin i.
    \item Estimator $p=n_i/n$
    \item Expect if you repeated $\Rightarrow$ $n_i$ would follow Binomial distribution with mean $\sigma_i = \sqrt{n p q}$ and $V_i = \sigma^2_i = np ( 1-p)$
    \item \[ p = \frac{n_i}{n} \]
    \item \[ \boxed{\sigma_i = \sqrt{n_i \left(1 - \frac{n_i}{n}\right)}} \approx \sqrt{n_i} \text{ if } n \gg n_i \]
    
    \item Notes: we did you know the total number n, how often is it in bin i.
    \item HW: given the distribution, how many times n do i need to do it to get that.
    \item r fixed n, vs n fixed r.
\end{itemize}
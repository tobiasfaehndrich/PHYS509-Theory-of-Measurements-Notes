\section[Heat Capacity Models III — Drude Model]{\hyperlink{toc}{Heat Capacity Models III — Drude Model}}

Recall:
\textbf{The Density of states}

\begin{itemize}
    \item It is conventional to express integrals of this type in terms of the density of states, $g(\omega)$, which is defined as the number of states per unit frequency interval, i.e.,
    \[ \avg{E} = \int_{0}^{\infty} \dd{\omega} \, g(\omega) \, \hbar \omega \brackets*{\frac{1}{e^{\beta \hbar \omega} - 1} + \frac{1}{2}} \]


    where $\beta = \frac{1}{k_B T}$, and $k_B$ is the Boltzmann constant.
    \[ g(\omega) = L^3 \frac{12 \pi \omega^2}{(2 \pi)^3 v^3} \]

    \item The density of states tells us how many new modes (sound waves here) are available at a given temperature / frequency ($\hbar \omega = k_B T$).

    \item The DOS for the Einstein model would look like a delta function at $\omega = \omega_E$.
    \item The DOS for the Debye model is a parabola ($g(\omega) \propto \omega^2$), which is a good approximation for low frequencies.

    \item Solving for the integral for $\avg{E}$ (as done in HW \#1) gives:
    
    \[ \avg{E} \propto \text{T-independent} + a \text{T}^4 \]

    where the T-independent term is the zero-point energy, and the $a \text{T}^4$ term is the contribution from the phonons. a being some constant.

    \item The low T limit accounts for the $T^3$ law
    
    \[ \dv{\avg{E}}{T} = C \propto a \text{T}^3 \]

    which agrees with experiment.

    \item The high T limit does not saturate at the Dulong-Petit Law value,
    
    \[ C \propto a \text{T}^3 \]

    still at all temperatures, and never saturates to the Dulong-Petit Law value of $3Nk_B$.
    
    \item Dybye comes up with a solution to this problem, imposing that there should only be as many modes as there are degrees of freedom, i.e. integrate between $0$ and $\omega_{\text{cutoff}}$
    
    \[ 3N = N_{\text{atoms}} \cdot 3 = \# \text{ allowed modes} \]
    
    This was built into Einstein's calculation.

    \[ 3 N = \int_{0}^{\omega_{\text{cutoff}}} \dd{\omega} \, g(\omega) \]

    then evaluate this integral to find the cutoff frequency, $\omega_{\text{cutoff}}$.

    \item This now gives the correct high T limit agreeing with the Dulong-Petit Law:
    \[ \avg{E} = \int_{0}^{\omega_{\text{cutoff}}} \dd{\omega} \, g(\omega) \, \hbar \omega \brackets*{\frac{1}{e^{\beta \hbar \omega} - 1} + \frac{1}{2}} \Rightarrow 3 R T \, \text{as} \, T \to \infty \]

    where $R$ is the gas constant, and $3 R$ is the Dulong-Petit Law value.

    \item This cutoff frequency is the Debye frequency, $\omega_D$, and the Debye temperature is defined as:

    \[ \omega_{\text{cutoff}} = \omega_D = \sqrt[3]{\frac{6 \pi^2 N^3}{V^3}}  = \sqrt[3]{6 \pi^2 n v^3}\]

    where $n$ is the number density of atoms ($n = \frac{N}{V}$), and $v$ is the speed of sound in the material.

    \item $T_D$ is the Debye temperature (also called $\theta_D$)
    \[ T_D = \frac{\hbar \omega_D}{k_B} \]
    
    \item If a material undergoes a phase transition causing its colume to half. What will happen to the Debye temperature? (assume $v$ is constant - is this a good assumption?)
    
    \begin{itemize}
        \item $T_D = \frac{\hbar \omega_D}{k_B} = \frac{\hbar}{k_B} \sqrt[3]{6 \pi^2 n v^3}$
        \item $n$ will double, so $T_D$ will increase by a factor of $\sqrt[3]{2}$.
        \item If the speed of sound is not constant, then the change in $T_D$ will depend on how the speed of sound changes with volume.
    \end{itemize}

    \item Debye and Einstein are similar, but Debye model is more accurate for low temperatures.
    
    \item Note that at intermediate temperatures, the integral in the Debye formulation can only be solved numerically.
    
    \item Debye model \underbar{very} successful, but not perfect:
    \begin{itemize}
        \item Less accurate at intermediate temperatures.
        \item Introduciton of $\omega_{\text{cutoff}}$ is somewhat arbitrary.
        \item Assumption of $\omega=vk$ becomes questionable for short wavelengths (high frequencies).
        \item Metals have an additional T-linear contribution to C that is missed by the Debye model.
    \end{itemize}
    
\end{itemize}


\subsection{Drude Theory (Chapter 3)}

\begin{itemize}
    \item So far, we've talked about atomic vibrations (effectively, a property of the nucleus). next we are going to talk about the conduction of electricity (a property of the electrons).
    \item What we already know from real life: some materials conduct electricity and are therefrore metals, and others do not. We are going to skip over (for now) what makes a material a metal and jump straight into understanding the properties of the conduction electrons.
    \item Basic idea: Drude (Drew-duh) extended Boltzmann's kinetic theory of gases (see the ideal gas law) to the motion of electrons in a metal (no atoms, just a gas of electrons).
\end{itemize}

\textbf{Assumptions of the Drude model}
\begin{itemize}
    \item The basic assumptions of the Drude model are:
    \begin{enumerate}
        \item The electrons have some characteristic scattering time, $\tau$ (material dependent, average time between collisions). The probability of scattering in a time interval $dt$ is $P = \frac{dt}{\tau}$. (reasonable. note that it assumes nothing about nature of scattering).
        \item After scattering, the electron has zero kinetic energy (i.e. $\vec{p} = 0$). This is questionable, but true on average.
        \item The electrons will respond to externally applied electric and magnetic fields. This is reasonable, electrons are charged.
    \end{enumerate}
\end{itemize}


\textbf{Equation of motion for the Drude Model}
\begin{itemize}
    \item In order to understand the behaviour of the conduction electrons, we will start by defining the average momentum for an electron, which depends on whether or not it experienced a scattering event.
    
    time $dt$ later: 
    \[
    \begin{aligned}
    \avg{\vec{p}(t+dt)} &= \Prob_{\text{no scatter}} \cdot \vec{p}_{\text{no scatter}} + \Prob_{\text{scatter}} \cdot \vec{p}_{\text{scatter}} \\
    &= \paren*{1-\frac{dt}{\tau}} \cdot \paren*{\vec{p}(t) + \vec{F}dt} + \paren*{\frac{dt}{\tau}} \cdot 0
    \end{aligned}
    \]

    where $d\vec{p} = \vec{F} dt $ and Lorentz force $\vec{F} = -e (\vec{E} + \vec{v} \times \vec{B})$

    \[ \avg{\vec{p}(t+dt)} = \vec{p}(t)-\frac{\vec{p}(t) dt}{\tau} + \vec{F}dt - \frac{\vec{F} dt^2}{\tau} \]


    \item For $dt \ll 1$, we can keep only terms that are of first-order in $dt$:
    
    We get an equation of motion!
    \[ \boxed{\frac{d\vec{p}}{d\tau} = \vec{F} - \frac{\vec{p}(t)}{\tau}} \]

    last term acts like a drag forc on the electron.
    
    \item What does this give in the absence of any applied field? Does that make sense?
    
    \[ \vec{F} = 0 \Rightarrow \frac{d\vec{p}}{d\tau} = -\frac{\vec{p}(t)}{\tau} \]

    \[ \vec{p}(t) = \vec{p}(0) \cdot \e^{-\frac{t}{\tau}} \]

    which gives an expenential decay of the momentum of the electron to zero. In the model this makes sense, but in real life this is not the case.
    
\end{itemize}









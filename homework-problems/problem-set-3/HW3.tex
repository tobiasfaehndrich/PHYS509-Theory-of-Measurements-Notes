\documentclass[11pt]{article}
\usepackage{commands}

% Document metadata
\title{PHYS 509 - Problem Set 3}
\author{Tobias Faehndrich}
\date{\today}

\begin{document}

\maketitle

\tableofcontents
\newpage

% ============================================================
% PROBLEM 1
% ============================================================
\section{Problem 1: Least Squares Fit of Exponential Decay (1000 Events)}

\begin{quote}
    File \texttt{exp1000.txt} contains a simulated list of decay times from an experiment.

    \textbf{(a)} Perform a least squares fit to this distribution for the lifetime $\tau$, to find the best estimate and also its uncertainty. Plot your fit results on your histogram of the data.

    \textbf{(b)} Plot a scan of the $\chi^2$ about the minimum over a range that shows at least the $\pm 3\sigma$ region with lines indicating the 1, 2 and 3$\sigma$ levels on the $\chi^2$ plot. Make a table with the symmetric and asymmetric confidence intervals for 1, 2 and 3$\sigma$ limits.
\end{quote}

\divider

\textbf{Tutor's Hint:}
\begin{itemize}
    \item For least squares fitting of histogram data, use $\chi^2 = \sum_i \frac{(n_i - \mu_i)^2}{\sigma_i^2}$ where $n_i$ is the observed count in bin $i$, and $\mu_i$ is the expected count.
    \item For Poisson statistics (counting), $\sigma_i^2 = n_i$ (variance equals mean).
    \item The exponential decay PDF is $p(t|\tau) = \frac{1}{\tau}e^{-t/\tau}$ for $t \geq 0$.
    \item Expected count in bin $i$ with edges $[t_{i-1}, t_i]$: $\mu_i = N \int_{t_{i-1}}^{t_i} p(t|\tau) dt = N(e^{-t_{i-1}/\tau} - e^{-t_i/\tau})$.
    \item Minimize $\chi^2$ to find best $\tau$. Uncertainty from $\Delta\chi^2 = 1$ (1 parameter).
    \item For confidence intervals: $\Delta\chi^2 = 1, 4, 9$ for 1, 2, 3$\sigma$ levels.
\end{itemize}



\subsection{Solution}


% ============================================================
% PROBLEM 2
% ============================================================
\section{Problem 2: Maximum Likelihood Fit of Exponential Decay (1000 Events)}

\begin{quote}
    Use the same file as above, but perform a maximum likelihood fit for $\tau$.

    \textbf{(a)} Perform a likelihood fit to this distribution for the lifetime $\tau$, to find the best estimate and also its uncertainty. Plot your fit result on a histogram of the data.

    \textbf{(b)} Plot a scan of the likelihood $L$ about the minimum over a range that shows at least the $\pm 3\sigma$ region with lines indicating the 1, 2 and 3$\sigma$ levels on the plot. Make a table with the symmetric and asymmetric confidence intervals for 1, 2 and 3$\sigma$ limits.
\end{quote}

\textbf{Tutor's Hint:}
\begin{itemize}
    \item For unbinned maximum likelihood, the likelihood function is $L(\tau) = \prod_{i=1}^{N} p(t_i|\tau)$ where $t_i$ are the individual measurements.
    \item It's easier to work with log-likelihood: $\ln L = \sum_{i=1}^{N} \ln p(t_i|\tau) = -N\ln\tau - \frac{1}{\tau}\sum_{i=1}^{N} t_i$.
    \item Maximize $\ln L$ (or minimize $-\ln L$) to find best $\tau$.
    \item The ML estimator for exponential lifetime: $\hat{\tau} = \bar{t} = \frac{1}{N}\sum_{i=1}^{N} t_i$ (sample mean).
    \item Uncertainty from $\Delta(-\ln L) = 0.5$ for 1 parameter (equivalent to $\Delta\chi^2 = 1$).
    \item For confidence intervals: $\Delta(-\ln L) = 0.5, 2, 4.5$ for 1, 2, 3$\sigma$ levels.
\end{itemize}

\subsection{Solution}


% ============================================================
% PROBLEM 3
% ============================================================
\section{Problem 3: Fits with Small Sample (30 Events)}

\begin{quote}
    Do the same fits as above, but with the file \texttt{exp30.txt}, which only has 30 entries.

    \textbf{(a)} Do the least squares fit on this file, with the same plots as above. Explain your histogram binning, how you determined it, and any other special treatment you had to make.

    \textbf{(b)} Do the ML fit, again with the same plots as above.
\end{quote}

\textbf{Tutor's Hint:}
\begin{itemize}
    \item With only 30 events, histogram binning becomes critical. Too many bins $\rightarrow$ empty bins, too few $\rightarrow$ loss of information.
    \item Rule of thumb: $N_{\text{bins}} \approx \sqrt{N} \approx 5-6$ bins for $N=30$.
    \item For Poisson statistics with small counts, use $\sigma_i^2 = n_i$ but be careful with empty bins ($n_i = 0$).
    \item Some options for handling empty bins: exclude from $\chi^2$, use $\sigma_i^2 = 1$, or use expected count $\mu_i$.
    \item ML fit doesn't require binning - works directly with unbinned data. Should be more reliable for small samples.
    \item Compare results: Which method gives tighter uncertainties? Which is more stable?
\end{itemize}

\subsection{Solution}


% ============================================================
% PROBLEM 4
% ============================================================
\section{Problem 4: Convolution of Exponential with Gaussian}

\begin{quote}
    Derive an expression for the \textbf{convolution} of an exponential decay with lifetime $\tau$, with a Gaussian resolution function. Assume the Gaussian is centered at 0, with variance $\sigma^2$. You should be able to express this function using an exponential and an erfc function.
\end{quote}

\textbf{Tutor's Hint:}
\begin{itemize}
    \item Convolution of two PDFs $f(t)$ and $g(t)$: $(f * g)(t) = \int_{-\infty}^{\infty} f(t') g(t - t') dt'$.
    \item Exponential: $f(t) = \frac{1}{\tau}e^{-t/\tau}$ for $t \geq 0$, zero otherwise.
    \item Gaussian: $g(t) = \frac{1}{\sqrt{2\pi\sigma^2}}e^{-t^2/(2\sigma^2)}$.
    \item The convolution integral: $h(t) = \int_{0}^{\infty} \frac{1}{\tau}e^{-t'/\tau} \cdot \frac{1}{\sqrt{2\pi\sigma^2}}e^{-(t-t')^2/(2\sigma^2)} dt'$.
    \item Trick: Complete the square in the exponent to get a Gaussian integral.
    \item Result involves $\text{erfc}(x) = \frac{2}{\sqrt{\pi}}\int_x^{\infty} e^{-u^2} du$ (complementary error function).
    \item Expected form: $h(t) = \frac{A}{\tau} e^{a + bt} \cdot \text{erfc}(c + dt)$ where you need to find $A, a, b, c, d$.
\end{itemize}

\subsection{Solution}


% ============================================================
% PROBLEM 5
% ============================================================
\section{Problem 5: ML Fit with Gaussian Smearing}

\begin{quote}
    Download the file \texttt{exp\_smeared.txt}, which contains a set of simulated data from an experiment measuring the lifetime of a particle, but with Gaussian smearing due to detector resolution. Perform a ML fit to the data to extract the lifetime $\tau$ and detector resolution $\sigma$, along with the full covariance matrix of these estimators. Use the Kolmogorov-Smirnov test to determine the goodness of fit for this model.

    Plot your fit result on top of a histogram of your data.
\end{quote}

\textbf{Tutor's Hint:}
\begin{itemize}
    \item The PDF is now the convolution result from Problem 4: $h(t|\tau, \sigma)$.
    \item Two parameters to fit: $\tau$ and $\sigma$. Use unbinned ML.
    \item Log-likelihood: $\ln L(\tau, \sigma) = \sum_{i=1}^{N} \ln h(t_i|\tau, \sigma)$.
    \item Covariance matrix from inverse Hessian: $V = H^{-1}$ where $H_{jk} = -\frac{\partial^2 \ln L}{\partial\theta_j \partial\theta_k}$.
    \item Diagonal elements $V_{jj} = \sigma_j^2$ (variances), off-diagonal $V_{jk}$ are covariances.
    \item Correlation: $\rho_{jk} = V_{jk}/(\sigma_j \sigma_k)$.
    \item Kolmogorov-Smirnov test: Compare empirical CDF with fitted CDF. $D = \max|F_{\text{data}}(t) - F_{\text{fit}}(t)|$.
    \item KS p-value tells you goodness of fit. $p > 0.05$ usually means acceptable fit.
\end{itemize}

\subsection{Solution}


% ============================================================
% PROBLEM 6
% ============================================================
\section{Problem 6: ML Fit with Signal and Background}

\begin{quote}
    Download the file \texttt{exp\_smear\_bckgnd.txt}, which contains data from a signal which is a smeared exponential similar to the one above, but also with a background of random events with uniform probability over the experiment's region of operation, which is from $t = -3$ to $t = 10$ seconds.

    The detector's time resolution is determined elsewhere to be a Gaussian $\mathcal{N}(0, \sigma^2)$ with $\sigma = 0.2$ s.

    Perform a ML fit to the data to extract the lifetime $\tau$, fraction of the sample that is signal, along with the full covariance matrix of these estimators.

    Plot your fit result on a histogram of the data, and also on the same plot show the fitted background and (smeared) signal fits.
\end{quote}

\textbf{Tutor's Hint:}
\begin{itemize}
    \item Combined PDF: $p(t|\tau, f) = f \cdot h(t|\tau, \sigma) + (1-f) \cdot p_{\text{bkg}}(t)$ where $f$ is signal fraction.
    \item Background PDF: $p_{\text{bkg}}(t) = \frac{1}{13}$ for $t \in [-3, 10]$, zero otherwise (uniform).
    \item Signal PDF: $h(t|\tau, \sigma)$ from Problem 4, with $\sigma = 0.2$ s (fixed).
    \item Two parameters: $\tau$ and $f$. Use unbinned ML.
    \item For plotting: $N_{\text{sig}} = f \cdot N_{\text{total}}$, $N_{\text{bkg}} = (1-f) \cdot N_{\text{total}}$.
    \item Signal component: $f \cdot h(t|\tau, \sigma) \cdot N_{\text{total}} \cdot \Delta t$ (for histogram).
    \item Background component: $(1-f) \cdot p_{\text{bkg}}(t) \cdot N_{\text{total}} \cdot \Delta t$.
    \item Total fit: sum of signal and background components.
\end{itemize}

\subsection{Solution}


% ============================================================
% PROBLEM 7
% ============================================================
\section{Problem 7: Gaussian Signal with Flat Background}

\begin{quote}
    File \texttt{gauss\_bckgnd.txt} contains the counts from an experiment with a Gaussian signal of unknown mean and sigma, along with a flat background. There are 50 bins, with edges listed in the file.

    \textbf{(a)} Do a least squares fit on the data for the amplitude, mean, and sigma of the signal ($S, \mu_0, \sigma_0$), and the background level $\mu_B$. The \textbf{significance} of your signal is defined as $S/\sigma_S$, where $\sigma_S^2$ is the variance of $S$ returned by your fit.

    \textbf{(b)} Plot the histogram of the data along with the total fit result, the signal component and the background.

    \textbf{(c)} Choose the bins that you consider contain the bulk of the signal based upon the fit. How many background events $B$ do you estimate are in this region, and what would you estimate is the statistical uncertainty $\sigma_B$ on this number. Explain your reasoning.

    \textbf{(d)} Given the number of signal events $S$ you obtained from your fit, what would the uncertainty $\sigma_S$ on this be if the data were background free? Compare this to what you actually obtained in the fit.

    \textbf{(e)} Considering all the bins you chose in your signal region, what is your estimate of the uncertainty on the size of statistical fluctuations in this region, expressed in terms of $S$ and $B$? What is your expected signal significance, then, and how does it compare with your fitted value?
\end{quote}

\textbf{Tutor's Hint:}
\begin{itemize}
    \item Model for bin counts: $\mu_i = S \cdot g_i(\mu_0, \sigma_0) + \mu_B$ where $g_i = \int_{x_{i-1}}^{x_i} \mathcal{N}(\mu_0, \sigma_0^2) dx$.
    \item Four parameters: $S$ (total signal events), $\mu_0$ (signal mean), $\sigma_0$ (signal width), $\mu_B$ (background per bin).
    \item $\chi^2 = \sum_i \frac{(n_i - \mu_i)^2}{n_i}$ where $n_i$ is observed count in bin $i$.
    \item Signal significance from fit: $S/\sigma_S$ where $\sigma_S$ comes from covariance matrix.
    \item For signal region (e.g., $[\mu_0 - 3\sigma_0, \mu_0 + 3\sigma_0]$): count bins in this range.
    \item Background in region: $B = \mu_B \cdot N_{\text{bins}}$, uncertainty $\sigma_B = \sqrt{\mu_B \cdot N_{\text{bins}}}$ (Poisson).
    \item If background-free: $\sigma_S = \sqrt{S}$ (Poisson).
    \item With background: total counts in region $N = S + B$, so $\sigma_N = \sqrt{N} = \sqrt{S + B}$.
    \item Signal significance: $\frac{S}{\sqrt{S + B}}$ (approximation when background is known).
\end{itemize}

\subsection{Solution}


% ============================================================
% PROBLEM 8
% ============================================================
\section{Problem 8: Measuring Gravitational Acceleration}

\begin{quote}
    In an undergraduate lab, students perform an experiment to measure $g$, by dropping a long clear plastic plate with a dark line marked every 10 cm over a 1 m length. The lines are sensed by a photosensor, and the time $t$ that each passes the sensor is recorded. The clock of the timer starts when the first line is sensed, so there are 11 measurements, starting with $t = 0$ for the line at 0 cm. The plate is held by hand some point above the photosensor before releasing.

    The file \texttt{gmeasure.txt} contains the times for the 11 markings, each with an uncertainty of 5 ms.

    \textbf{(a)} What is the equation for the position of the plate $z(t)$, where $z =$ vertical position of the 0 cm line marking wrt the photosensor position? (For uniformity, let's take $z$ increasing the downward direction.)

    \textbf{(b)} Since your experimentally measured variable is $t$, invert this so $t$ is the dependent variable and then fit for $g$. Plot the data, with errorbars, along with your fit result.

    \textbf{(c)} Instead, treat $z$ as the dependent variable, and fit for $g$ using the original equation you had. You will need to convert the uncertainty on $t$ to an approximate uncertainty on $z$ at each point that will change as your fit parameters change. Plot the data, with errorbars, along with your fit result. Compare with the previous results.
\end{quote}

\textbf{Tutor's Hint:}
\begin{itemize}
    \item Free fall from rest at initial position $z_0$ above sensor: $z(t) = z_0 + \frac{1}{2}gt^2$.
    \item At $t=0$, the 0 cm line is at the sensor, so $z(0) = 0 \Rightarrow z_0 = 0$.
    \item Actually, the plate was held above and released. When does the 0 cm line reach the sensor?
    \item More carefully: If released at height $h$ above sensor at $t_{\text{release}}$, then at sensor crossing: $0 = -h + \frac{1}{2}g(t_0)^2$.
    \item Redefine clock: $t = 0$ when 0 cm line passes sensor. Then for subsequent lines: $z_i = \frac{1}{2}gt_i^2 + v_0 t_i$ where $v_0$ is velocity at $t=0$.
    \item Simplification: If we take $z$ as position of each line wrt 0 cm line, then $z_i = i \cdot 10$ cm for line $i = 0, 1, 2, \ldots, 10$.
    \item For part (b): $t = \sqrt{\frac{2z}{g}}$ (if starting from rest). Fit $t$ vs $z$ data.
    \item For part (c): $z = \frac{1}{2}gt^2$. Error propagation: $\sigma_z = \left|\frac{dz}{dt}\right| \sigma_t = gt \cdot \sigma_t$.
    \item Note: uncertainty on $z$ depends on $g$ (the parameter you're fitting!). Iterate: fit, recalculate errors, refit.
\end{itemize}

\subsection{Solution}


\end{document}
